\documentclass[12pt,a4paper]{article}

% --- Idioma y codificación ---
\usepackage[spanish]{babel}
\usepackage[utf8]{inputenc}   % si compilas con pdflatex
\usepackage[T1]{fontenc}
\usepackage{titlesec}
\titleformat{\section}
  {\normalfont\large\bfseries}
  {\thesection}{1em}{}
\titleformat{\subsection}
  {\normalfont\normalsize\bfseries}
  {\thesubsection}{1em}{}

% --- Márgenes (ajústalos a tu gusto) ---
\usepackage[a4paper,
  top=2.5cm,bottom=2.5cm,
  left=2.2cm,right=2.2cm,
  columnsep=0.7cm % separación entre columnas
]{geometry}


% --- Gráficos, tablas, etc. ---
\usepackage{graphicx}
\usepackage{subfig}
\usepackage{physics}
\usepackage{csquotes}
\usepackage{float}

% --- Dos columnas tipo paper (controlable por entorno) ---
\usepackage{multicol}

% --- Índice y enlaces ---
\usepackage[hidelinks]{hyperref}
\addto\captionsspanish{\renewcommand{\contentsname}{Índice}}

% --- Encabezados y pies (opcional, estilo paper) ---
\usepackage{fancyhdr}
\pagestyle{fancy}
\fancyhf{}
\lhead{Péndulo Simple}
\rhead{\leftmark}
\cfoot{\thepage}
\setlength{\headheight}{14.5pt}

% --- Bibliografía (usa SOLO una vez) ---
\usepackage[backend=biber,style=ieee]{biblatex}
\addbibresource{bibliografia.bib}

% --- Para que no “salte” a páginas pares tipo libro ---
\let\cleardoublepage\clearpage
\raggedcolumns

\begin{document}

% =========================
% Portada (1 columna)
% =========================
\begin{titlepage}
  \centering
  \vspace*{2cm}
  {\LARGE \textbf{Péndulo Simple}\par}
  \vspace{1cm}
  {\large Asignatura: Física I\par}
  \vspace{0.5cm}
  {\large Autor: Luis López Nasser\par}
  {\large Fecha: 18/12/2025\par}
  \vfill
  {\large Universidad Unie\par}
\end{titlepage}

% =========================
% Índice (1 columna)
% =========================
\tableofcontents
\clearpage

% =========================
% Contenido (2 columnas)
% =========================
\begin{multicols}{2}

\section{Introducción}

El péndulo simple es un sistema físico idealizado formado por una masa puntual suspendida de un hilo inextensible y de masa despreciable, que oscila bajo la acción del campo gravitatorio. A pesar de su simplicidad, constituye un modelo fundamental para el estudio de los movimientos oscilatorios y permite analizar con precisión el movimiento armónico simple en determinadas condiciones.

\bigskip
Cuando el péndulo se separa ligeramente de su posición de equilibrio y se libera, la masa comienza a oscilar describiendo un movimiento periódico. La variable relevante para describir el sistema es el ángulo $\theta(t)$ que forma el hilo con la vertical. La fuerza responsable del movimiento es la componente tangencial del peso, que actúa como fuerza restauradora y tiende a devolver la masa a su posición de equilibrio.

\begin{figure}[H]
\centering
\includegraphics[width=0.7\columnwidth]{imagenes/esquema_pendulo}
\caption{Esquema del péndulo simple}
\label{fig:pendulo}
\end{figure}

Para ángulos de oscilación pequeños, puede aplicarse la aproximación $\sin\theta \approx \theta$, lo que permite linealizar la ecuación de movimiento. Bajo esta hipótesis, el sistema se comporta como un oscilador armónico simple, cuya ecuación diferencial conduce a una solución periódica con un período independiente de la masa del cuerpo oscilante. En este régimen, el período de oscilación del péndulo viene dado únicamente por la longitud del hilo y la aceleración de la gravedad.

\bigskip
Esta relación teórica constituye la base del experimento, ya que permite determinar experimentalmente el valor de la aceleración de la gravedad a partir de la medida del período para distintas longitudes del péndulo. Además, el estudio del período en función de la longitud permite verificar la validez del modelo armónico y su dependencia funcional.

\subsection{Marco teórico}

Un movimiento armónico simple (MAS) es aquel caracterizado por la acción de una fuerza recuperadora proporcional y opuesta al desplazamiento respecto a la posición de equilibrio. Dicha fuerza puede expresarse como
\begin{equation}
F=-kx,
\end{equation}
donde $k$ es la constante recuperadora del sistema y $x$ el desplazamiento.

\bigskip
Aplicando la segunda ley de Newton, se obtiene una ecuación diferencial de segundo grado para la posición en función del tiempo:
\begin{equation}
\frac{d^{2}x(t)}{dt^{2}}+\omega^{2}x(t)=0,
\end{equation}
donde,
\begin{equation}
\omega=\sqrt{\frac{k}{m}}
\end{equation}
es la frecuencia angular natural del sistema. La solución general de esta ecuación es una función periódica del tiempo que puede escribirse como,
\begin{equation}
x(t)=A\sin(\omega t+\phi),
\end{equation}
siendo $A$ la amplitud del movimiento y $\phi$ la fase inicial. El período del movimiento armónico simple está relacionado con la frecuencia angular mediante,
\begin{equation}
T=\frac{2\pi}{\omega}.
\end{equation}

En el caso de un péndulo simple, el sistema está formado por una masa puntual suspendida de un hilo inextensible y de masa despreciable. La variable cinemática relevante es el ángulo $\theta(t)$ que forma el péndulo con respecto a la vertical. La componente tangencial del peso, responsable del movimiento oscilatorio, viene dada por,
\begin{equation}
P_x=-mg\sin\theta.
\end{equation}

Aplicando la segunda ley de Newton y teniendo en cuenta la relación entre el desplazamiento lineal y el ángulo, $x=L\theta$, se obtiene la ecuación diferencial que describe el movimiento del péndulo:
\begin{equation}
\frac{d^{2}\theta(t)}{dt^{2}}+\frac{g}{L}\sin\theta(t)=0.
\end{equation}

Esta ecuación es no lineal y no admite una solución analítica sencilla. Sin embargo, en el límite de ángulos pequeños, puede aplicarse la aproximación,
\begin{equation}
\sin\theta \approx \theta,
\end{equation}
con lo que la ecuación de movimiento se linealiza y describe un movimiento armónico simple. En este régimen, la frecuencia angular del péndulo viene dada por,
\begin{equation}
\omega=\sqrt{\frac{g}{L}},
\end{equation}
y el período de oscilación resulta,
\begin{equation}
T=2\pi\sqrt{\frac{L}{g}}.
\end{equation}

Esta expresión muestra que, para pequeñas amplitudes, el período del péndulo simple depende únicamente de la longitud del hilo y de la aceleración de la gravedad, siendo independiente de la masa del cuerpo oscilante. Esta relación constituye la base del estudio experimental realizado en la práctica.

\bigskip

En resumen, el péndulo simple permite estudiar de forma experimental el movimiento armónico simple y contrastar sus predicciones teóricas. La expresión del período obtenida bajo la aproximación de ángulos pequeños constituye la base para el análisis de los resultados y la determinación de la aceleración de la gravedad.

\columnbreak

\subsection{Objetivos}

El objetivo principal de esta práctica es estudiar experimentalmente el comportamiento del péndulo simple y analizar su movimiento oscilatorio en el régimen de pequeñas amplitudes. A partir de la medida del período de oscilación para distintas longitudes del hilo, se pretende comprobar la validez del modelo teórico que describe al péndulo como un movimiento armónico simple y analizar la relación existente entre el período y la longitud del sistema.

\bigskip
Asimismo, se busca verificar experimentalmente la independencia del período de oscilación respecto a la masa del cuerpo suspendido y determinar un valor experimental de la aceleración de la gravedad a partir de los datos obtenidos. De manera complementaria, se estudiará la influencia de la amplitud angular inicial sobre el período de oscilación, con el fin de identificar las posibles desviaciones respecto al modelo ideal y evaluar los límites de validez de la aproximación armónica.

\vspace{3 em}

\begin{figure}[H]
\centering
\includegraphics[width=0.9\columnwidth]{imagenes/foto_pendulo}
\caption{Péndulo simple}
\label{fig:foto_pendulo}
\end{figure}

\newpage

\section{Materiales}

Para la realización de la práctica del péndulo simple se empleó un conjunto de materiales sencillos, seleccionados con el fin de garantizar un montaje experimental estable y permitir la realización de medidas precisas del período de oscilación y del ángulo inicial. El uso adecuado de estos elementos resulta fundamental para minimizar errores experimentales y asegurar la reproducibilidad de las medidas.

\bigskip
Los materiales utilizados fueron los siguientes:
\begin{itemize}
    \item Hilo de longitud variable.
    \item Pesa metálica.
    \item Cronómetro.
    \item Soporte.
    \item Transportador de ángulos.
\end{itemize}

\section{Procedimiento experimental}

El experimento se llevó a cabo con el objetivo de estudiar el comportamiento del péndulo simple y verificar experimentalmente las expresiones teóricas que describen su período de oscilación. Para ello, se realizó un conjunto de medidas en las que se varió de forma sistemática la longitud del hilo que sostenía la masa oscilante.

\bigskip
Para una determinada longitud del hilo y con una amplitud angular pequeña, se hizo oscilar la pesa y se midió el tiempo necesario para que el péndulo completara varias oscilaciones consecutivas, típicamente entre cinco y diez. A partir del tiempo total medido y del número de oscilaciones realizadas, se obtuvo el valor del período medio dividiendo ambos valores. Este procedimiento permite reducir la influencia de los errores asociados a la medida del tiempo.

\bigskip
El proceso se repitió para distintos valores de la longitud del hilo, razonablemente diferentes entre sí, obteniendo el período de oscilación correspondiente a cada longitud. Con el fin de mejorar la fiabilidad de los resultados, cada medida se repitió varias veces bajo las mismas condiciones experimentales, trabajando posteriormente con los valores promedio obtenidos.

\bigskip
Además del estudio de la dependencia del período con la longitud, se analizó la posible influencia de otras variables. Para algunas de las longitudes consideradas, se repitieron las medidas utilizando distintas masas suspendidas, con el objetivo de comprobar la independencia del período respecto a la masa. Asimismo, se realizaron medidas con diferentes amplitudes angulares iniciales, empleando valores mayores del ángulo de lanzamiento, con el fin de estudiar la aparición de efectos anarmónicos y analizar las desviaciones respecto al modelo armónico ideal.

\bigskip
Finalmente, los períodos obtenidos experimentalmente se compararon con las expresiones teóricas del péndulo simple, tanto en el régimen de pequeñas amplitudes como considerando las correcciones asociadas a oscilaciones de mayor amplitud, lo que permitió evaluar el grado de concordancia entre la teoría y los resultados experimentales.


\newpage

\section{Datos experimentales}

En esta sección se presentan los datos obtenidos durante la realización del experimento del péndulo simple. A partir de las medidas del tiempo de oscilación para distintas longitudes del hilo, masas suspendidas y amplitudes angulares iniciales, se calcularon los períodos de oscilación correspondientes. Estos datos se muestran a continuación mediante tablas y representaciones gráficas, que permiten visualizar las tendencias experimentales y facilitan su posterior análisis y comparación con las predicciones teóricas.

\begin{figure}[H]
\centering
\includegraphics[width=1\columnwidth]{imagenes/datos_pendulo}
\caption{Datos experimentales}
\label{fig:datos_pendulo}
\end{figure}

A partir de los datos recogidos se observa una variación del período de oscilación con la longitud del péndulo, así como una ligera dependencia con la amplitud angular inicial para valores elevados del ángulo. 

\bigskip
Antes de proceder al análisis de los resultados experimentales, conviene señalar que el período de oscilación de un péndulo simple es independiente de la masa del cuerpo oscilante desde el punto de vista teórico, como se obserba en la ecuacion \eqref{eq:periodo}. Y puede comprobarse tanto desde el punto de vista experimental, al observar que distintas masas dan lugar a períodos prácticamente iguales, como se observa en la siguiente gráfica:

\begin{figure}[H]
\centering
\includegraphics[width=1\columnwidth]{imagenes/perido_masa}
\caption{Período en función de la masa}
\label{fig:periodo_masa}
\end{figure}

Dado que la masa no interviene en el período, se concluye que este depende únicamente de la longitud del hilo y de la aceleración de la gravedad. Sobre esta base, se procederá al análisis de los datos experimentales con el fin de estudiar la relación entre el período de oscilación, la longitud del péndulo y el ángulo inicial.

\section{Analisis y Discusión de datos}

\bigskip
A continuación se representa gráficamente el cuadrado del período de oscilación en función de la longitud del péndulo. De acuerdo con la expresión teórica correspondiente, se espera que los datos experimentales se ajusten a una relación lineal. Mediante un ajuste por mínimos cuadrados se obtiene la recta que mejor describe los datos, cuya pendiente permite determinar un valor experimental de la aceleración de la gravedad.

\bigskip
\noindent
Para $\theta = 30^\circ$:
\vspace{-0.4 cm}
\begin{figure}[H]
\centering
\includegraphics[width=1\columnwidth]{imagenes/primera}
\caption{ }
\label{fig:primera}
\end{figure}

\columnbreak

\noindent
Para $\theta = 60^\circ$:
\vspace{-0.4 cm}
\begin{figure}[H]
\centering
\includegraphics[width=1\columnwidth]{imagenes/segunda}
\caption{ }
\label{fig:segunda}
\end{figure}

\noindent
Para $\theta = 90^\circ$:
\vspace{-0.4 cm}
\begin{figure}[H]
\centering
\includegraphics[width=1\columnwidth]{imagenes/tercera}
\caption{ }
\label{fig:tercera}
\end{figure}

Al analizar las representaciones gráficas de $T^2$ en función de la longitud del péndulo para los distintos valores del ángulo inicial, se observa que los valores de la aceleración de la gravedad obtenidos difieren entre sí. En particular, para cada ángulo de oscilación se obtiene una pendiente distinta en el ajuste lineal, lo que conduce a valores experimentales de $g$ diferentes.

\bigskip
La determinación de la aceleración de la gravedad se ha realizado a partir de la pendiente de la recta ajustada a los datos experimentales. De acuerdo con el modelo teórico del péndulo simple en el régimen de pequeñas amplitudes,
\begin{equation}
T^2 = \frac{4\pi^2}{g}\,L,
\end{equation}
por lo que la pendiente de la recta en la representación $T^2$ frente a $L$ viene dada por $b = 4\pi^2/g$. A partir de esta relación, el valor experimental de la aceleración de la gravedad se obtiene como $g = 4\pi^2/b$, propagando la incertidumbre asociada a la pendiente del ajuste.

\bigskip
Sin embargo, los valores de $g$ obtenidos resultan significativamente inferiores al valor esperado de la aceleración de la gravedad. Esta discrepancia puede atribuirse a varios factores. En primer lugar, la expresión teórica utilizada es válida únicamente en el límite de ángulos pequeños. Para ángulos grandes, como $60^\circ$ y $90^\circ$, el movimiento deja de ser estrictamente armónico simple y el período de oscilación aumenta respecto al valor predicho por el modelo ideal. Este aumento del período provoca una mayor pendiente en la gráfica $T^2$ frente a $L$, lo que conduce a una subestimación del valor de $g$.

\bigskip
Además, pueden intervenir otros factores experimentales, como errores sistemáticos en la medida del tiempo y de la longitud, la dificultad para definir con precisión el instante de paso por la posición de equilibrio, o la influencia del rozamiento con el aire y de la fricción en el punto de suspensión, que tienden a alargar el período de oscilación. Todos estos efectos contribuyen a que el valor experimental de la aceleración de la gravedad resulte inferior al valor teórico, siendo esta discrepancia más acusada cuanto mayor es la amplitud angular inicial.

\bigskip
Con el fin de profundizar en el análisis y comprobar de forma alternativa la validez de la relación teórica entre el período de oscilación y la longitud del péndulo, se repite la representación de los datos en escala logarítmica. En esta representación, la pendiente del ajuste lineal permite verificar la dependencia funcional entre el período y la longitud predicha por la ecuación~\eqref{eq:periodo}, mientras que la ordenada en el origen proporciona un valor experimental de la aceleración de la gravedad.

\newpage

\noindent
Para $\theta = 30^\circ$:
\vspace{-0.4 cm}
\begin{figure}[H]
\centering
\includegraphics[width=1\columnwidth]{imagenes/cuarta}
\caption{ }
\label{fig:cuarta}
\end{figure}

\noindent
Para $\theta = 60^\circ$:
\vspace{-0.4 cm}
\begin{figure}[H]
\centering
\includegraphics[width=1\columnwidth]{imagenes/quinta}
\caption{ }
\label{fig:quinta}
\end{figure}

\noindent
Para $\theta = 90^\circ$:
\vspace{-0.4 cm}
\begin{figure}[H]
\centering
\includegraphics[width=1\columnwidth]{imagenes/sexta}
\caption{ }
\label{fig:sexta}
\end{figure}

En las Figuras correspondientes a la representación logarítmica se ha representado $\ln(T)$ en función de $\ln(L)$ para $\theta = 30^\circ$, $60^\circ$ y $90^\circ$. Esta representación es útil porque, a partir de la expresión teórica del péndulo simple en el régimen armónico (10), puede escribirse en forma lineal aplicando logaritmos:

\columnbreak

\begin{equation}
\ln(T) = \ln(2\pi) + \frac{1}{2}\ln(L) - \frac{1}{2}\ln(g).
\end{equation}

De este modo, si se ajusta experimentalmente una recta del tipo
\begin{equation}
\ln(T) = c_0 + n\,\ln(L),
\end{equation}

la pendiente $n$ debe aproximarse a $1/2$ si la dependencia funcional $T\propto \sqrt{L}$ se cumple. En los resultados obtenidos, las pendientes ajustadas se encuentran próximas a $0.5$, lo cual indica que la relación funcional entre período y longitud es coherente con el modelo ideal.

\bigskip
La aceleración de la gravedad se obtiene a partir de la ordenada en el origen $c_0$. 
\begin{equation}
c_0 = \ln(2\pi) - \frac{1}{2}\ln(g),
\end{equation}
y, por tanto,
\begin{equation}
g = (2\pi)^2\,e^{-2c_0}.
\end{equation}
La incertidumbre de $g$ se ha calculado por propagación a partir del error del ajuste en $c_0$, de modo que
\begin{equation}
\Delta g = \left|\frac{dg}{dc_0}\right|\Delta c_0 = 2g\,\Delta c_0.
\end{equation}

A pesar de que la pendiente es compatible con el valor teórico esperado, los valores obtenidos de $g$ resultan significativamente inferiores al valor estándar ($g \approx 9.8\,\mathrm{m/s^2}$). Este hecho sugiere la presencia de un error sistemático dominante. La causa más probable es una discrepancia en la longitud efectiva utilizada: si la longitud $L$ introducida en el análisis es menor que la longitud real (por ejemplo, por usar una \emph{altura} o una distancia parcial en lugar de la distancia desde el punto de suspensión hasta el centro de masa), el ajuste produce una pendiente compatible con la tendencia $T\propto \sqrt{L}$, pero con un valor de $g$ artificialmente bajo. De hecho, para longitudes del orden de $0.09$--$0.15\,\mathrm{m}$ el modelo predice períodos claramente menores que los medidos; este desfase se traduce directamente en una subestimación de $g$ al despejarlo en la ecuación del período.

\bigskip
Además del posible error en la longitud efectiva, existen otros factores que tienden a aumentar el período medido y, por tanto, a reducir el valor estimado de $g$. Entre ellos destacan el tiempo de reacción al cronometrar, la fricción en el punto de suspensión, el rozamiento con el aire y el hecho de que para ángulos grandes ($60^\circ$ y $90^\circ$) el movimiento deja de ser estrictamente armónico simple, aumentando ligeramente el período respecto al valor ideal. No obstante, estas correcciones por amplitud y pérdidas disipativas suelen ser pequeñas frente a una discrepancia tan acusada, por lo que el efecto principal apunta a un error sistemático en la determinación de $L$ (o en su conversión/unidades) más que a una limitación intrínseca del modelo.

\bigskip
En conjunto, la representación logarítmica confirma la dependencia funcional esperada (pendiente cercana a $1/2$), pero también pone de manifiesto que la determinación absoluta de $g$ es muy sensible a errores sistemáticos en la longitud efectiva y en la medida del período, lo cual explica la desviación observada en los valores experimentales.

\bigskip
Una vez analizada la dependencia del período de oscilación con la longitud del péndulo y evaluada la validez del modelo teórico tanto en escala lineal como logarítmica, se procede a estudiar la influencia de la amplitud angular inicial en el período. Para ello, se representan gráficamente los períodos de oscilación en función del ángulo inicial, incluyendo en una misma gráfica las medidas correspondientes a las distintas longitudes analizadas. Este análisis permite comprobar si el período permanece constante al variar la amplitud angular y discutir las posibles desviaciones respecto al comportamiento ideal, así como su origen físico.

\begin{figure}[H]
\centering
\includegraphics[width=1\columnwidth]{imagenes/ultima}
\caption{ }
\label{fig:ultima}
\end{figure}

En la Figura correspondiente se representa el período de oscilación en función de la amplitud angular inicial para las distintas longitudes analizadas. En una misma gráfica se han incluido todas las medidas experimentales, lo que permite estudiar de forma global la influencia del ángulo inicial sobre el período del péndulo.

\bigskip
A partir de los datos experimentales se observa que, para un mismo valor de la longitud, el período de oscilación aumenta ligeramente al incrementar la amplitud angular. Esta tendencia es común a todas las longitudes consideradas, lo que indica que el efecto observado no depende de la longitud del hilo, sino de la amplitud del desplazamiento angular inicial.

\bigskip
En el modelo ideal del péndulo simple, válido únicamente en el límite de ángulos pequeños, el período de oscilación viene dado por la ecuación (10), expresión que no depende de la amplitud angular. Sin embargo, cuando el ángulo inicial deja de ser pequeño, esta aproximación deja de ser válida. Para amplitudes grandes, el movimiento ya no es estrictamente armónico simple y el período real resulta mayor que el predicho por el modelo ideal. Este comportamiento se debe a que, a grandes ángulos, la componente restauradora de la fuerza gravitatoria deja de ser proporcional al desplazamiento angular, lo que ralentiza el movimiento y aumenta el tiempo necesario para completar una oscilación.

Por tanto, los resultados experimentales muestran que el período sí varía con la amplitud angular, aumentando a medida que el ángulo inicial es mayor. Este efecto se hace especialmente evidente al comparar los valores correspondientes a $30^\circ$, $60^\circ$ y $90^\circ$. La coherencia de esta tendencia para todas las longitudes analizadas refuerza la interpretación física del fenómeno y pone de manifiesto una de las principales limitaciones del modelo del péndulo simple en su forma más básica.

\bigskip
En consecuencia, el análisis confirma que la independencia del período respecto a la amplitud angular solo es válida en el régimen de pequeñas oscilaciones, mientras que para ángulos grandes deben considerarse correcciones anarmónicas para describir correctamente el movimiento.

\section{Conclusiones}

En esta práctica se ha estudiado experimentalmente el comportamiento del péndulo simple con el objetivo de analizar la dependencia de su período de oscilación con la longitud del hilo y con la amplitud angular inicial, así como de evaluar la validez del modelo teórico en diferentes regímenes de oscilación.

\bigskip
En primer lugar, a partir de la representación del cuadrado del período en función de la longitud del péndulo y del correspondiente ajuste lineal por mínimos cuadrados, se ha comprobado que los datos experimentales siguen una relación aproximadamente lineal, en concordancia con la expresión teórica del período del péndulo simple para pequeñas amplitudes. Este resultado confirma la dependencia funcional $T \propto \sqrt{L}$ predicha por el modelo ideal.

\bigskip
La determinación experimental de la aceleración de la gravedad se ha realizado a partir de la pendiente de dichos ajustes, así como mediante la ordenada en el origen en la representación logarítmica $\ln(T)$ frente a $\ln(L)$. En ambos casos, aunque la tendencia funcional observada es coherente con la teoría, los valores obtenidos de $g$ resultan significativamente inferiores al valor tabulado. Esta discrepancia pone de manifiesto la fuerte sensibilidad del método a errores sistemáticos, especialmente en la determinación de la longitud efectiva del péndulo, así como a desviaciones del régimen ideal debido a rozamientos, fricción en el punto de suspensión y limitaciones en la medida del tiempo.

\bigskip
El análisis en escala logarítmica ha permitido, además, verificar de manera directa la relación de tipo potencial entre el período y la longitud. Las pendientes obtenidas en estas representaciones se encuentran próximas al valor teórico esperado de $1/2$, lo que refuerza la validez de la dependencia funcional, independientemente de la discrepancia en el valor absoluto de la aceleración de la gravedad.

\bigskip
Por otro lado, el estudio de la dependencia del período con la amplitud angular inicial ha mostrado que el período no permanece constante cuando se incrementa el ángulo de oscilación. Los resultados experimentales evidencian un aumento progresivo del período al pasar de $30^\circ$ a $60^\circ$ y $90^\circ$, para todas las longitudes analizadas. Este comportamiento confirma que la independencia del período respecto a la amplitud solo es válida en el límite de pequeñas oscilaciones, y que para ángulos grandes el movimiento deja de ser estrictamente armónico simple, siendo necesario considerar correcciones anarmónicas.

\bigskip
En conjunto, los resultados obtenidos permiten concluir que el péndulo simple constituye un modelo adecuado para describir el movimiento oscilatorio únicamente bajo condiciones ideales y de pequeñas amplitudes. La práctica ha puesto de manifiesto tanto la utilidad del modelo teórico como sus limitaciones, subrayando la importancia de los errores sistemáticos y de las aproximaciones empleadas en la interpretación de los datos experimentales.

\newpage

falta corregir lo que es el texto, mirar que esté todo bien, los links a las referencias de las ecuaciones añadir que se puede hacer para corregir la práctica y que de unos mejores valores
si eso poner el logo de la unie en la portada y algo mas que ahora no caigo

\end{multicols}

\end{document}
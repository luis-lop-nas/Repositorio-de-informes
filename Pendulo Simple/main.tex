\documentclass[12pt,a4paper]{article}

% --- Idioma y codificación ---
\usepackage[spanish]{babel}
\usepackage[utf8]{inputenc}   % si compilas con pdflatex
\usepackage[T1]{fontenc}

% --- Márgenes (ajústalos a tu gusto) ---
\usepackage[a4paper,
  top=2.5cm,bottom=2.5cm,
  left=2.2cm,right=2.2cm,
  columnsep=0.7cm % separación entre columnas
]{geometry}

% --- Gráficos, tablas, etc. ---
\usepackage{graphicx}
\usepackage{subfig}
\usepackage{physics}
\usepackage{csquotes}

% --- Dos columnas tipo paper (controlable por entorno) ---
\usepackage{multicol}

% --- Índice y enlaces ---
\usepackage[hidelinks]{hyperref}
\addto\captionsspanish{\renewcommand{\contentsname}{Índice}}

% --- Encabezados y pies (opcional, estilo paper) ---
\usepackage{fancyhdr}
\pagestyle{fancy}
\fancyhf{}
\lhead{Informe de Física}
\rhead{\leftmark}
\cfoot{\thepage}
\setlength{\headheight}{14.5pt}

% --- Bibliografía (usa SOLO una vez) ---
\usepackage[backend=biber,style=ieee]{biblatex}
\addbibresource{bibliografia.bib}

% --- Para que no “salte” a páginas pares tipo libro ---
\let\cleardoublepage\clearpage

\begin{document}

% =========================
% Portada (1 columna)
% =========================
\begin{titlepage}
  \centering
  \vspace*{2cm}
  {\LARGE \textbf{Péndulo Simple}\par}
  \vspace{1cm}
  {\large Asignatura: Física I\par}
  \vspace{0.5cm}
  {\large Autor: Luis López Nasser\par}
  {\large Fecha: 18/12/2025\par}
  \vfill
  {\large Universidad / Universidad Unie\par}
\end{titlepage}

% =========================
% Índice (1 columna)
% =========================
\tableofcontents
\clearpage

% =========================
% Contenido (2 columnas)
% =========================
\begin{multicols}{2}

\section{Introducción}
Texto de contexto...

\section{Marco teórico}
Fundamentos, ecuaciones, conceptos...

\section{Objetivos}
\begin{itemize}
  \item Objetivo 1...
  \item Objetivo 2...
\end{itemize}

\section{Procedimiento experimental}
Montaje, materiales, pasos...

\section{Datos experimentales}
Tablas, medidas, condiciones...

\section{Cálculos y tratamiento de datos}
Despejes, ajustes, gráficas, etc...

\section{Análisis de errores e incertidumbres}
Propagación de incertidumbres, error relativo, etc...

\section{Discusión de resultados}
Interpretación física, comparación con teoría, limitaciones...

\section{Conclusiones}
Puntos clave y cierre...

\end{multicols}

% Bibliografía (normalmente va a 1 columna para legibilidad; si la quieres a 2, quita multicols antes)
\printbibliography

\end{document}
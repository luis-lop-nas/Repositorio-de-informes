\documentclass[12pt,a4paper]{article}

% --- Idioma y codificación ---
\usepackage[spanish]{babel}
\usepackage[utf8]{inputenc}   % si compilas con pdflatex
\usepackage[T1]{fontenc}
\usepackage{titlesec}
\titleformat{\section}
  {\normalfont\large\bfseries}
  {\thesection}{1em}{}
\titleformat{\subsection}
  {\normalfont\normalsize\bfseries}
  {\thesubsection}{1em}{}

% --- Márgenes (ajústalos a tu gusto) ---
\usepackage[a4paper,
  top=2.5cm,bottom=2.5cm,
  left=2.2cm,right=2.2cm,
  columnsep=0.7cm % separación entre columnas
]{geometry}


% --- Gráficos, tablas, etc. ---
\usepackage{graphicx}
\usepackage{subfig}
\usepackage{physics}
\usepackage{csquotes}
\usepackage{float}

% --- Dos columnas tipo paper (controlable por entorno) ---
\usepackage{multicol}

% --- Índice y enlaces ---
\usepackage[hidelinks]{hyperref}
\addto\captionsspanish{\renewcommand{\contentsname}{Índice}}

% --- Encabezados y pies (opcional, estilo paper) ---
\usepackage{fancyhdr}
\pagestyle{fancy}
\fancyhf{}
\lhead{Rueda de Maxwell}
\rhead{\leftmark}
\cfoot{\thepage}
\setlength{\headheight}{14.5pt}

% --- Bibliografía (usa SOLO una vez) ---
\usepackage[backend=biber,style=ieee]{biblatex}
\addbibresource{bibliografia.bib}

% --- Para que no “salte” a páginas pares tipo libro ---
\let\cleardoublepage\clearpage
\raggedcolumns

\begin{document}

% =========================
% Portada (1 columna)
% =========================
\begin{titlepage}
  \centering
  \vspace*{2cm}
  {\LARGE \textbf{Rueda de Maxwell}\par}
  \vspace{1cm}
  {\large Asignatura: Física I\par}
  \vspace{0.5cm}
  {\large Autor: Luis López Nasser\par}
  {\large Fecha: 18/12/2025\par}
  \vfill
  {\large Universidad Unie\par}
\end{titlepage}

% =========================
% Índice (1 columna)
% =========================
\tableofcontents
\clearpage

% =========================
% Contenido (2 columnas)
% =========================
\begin{multicols}{2}

\section{Introducción}

La rueda de Maxwell es un dispositivo experimental clásico utilizado para el estudio del movimiento combinado de traslación y rotación en un sólido rígido. Consiste en una rueda o disco suspendido mediante uno o varios hilos enrollados alrededor de su eje. Al soltarse desde el reposo, la rueda desciende desenrollando los hilos mientras gira sobre su propio eje, de modo que la energía potencial gravitatoria se transforma simultáneamente en energía cinética de traslación y de rotación.

\bigskip
Este sistema se emplea habitualmente en prácticas de laboratorio de física porque permite ilustrar de forma clara y cuantitativa principios fundamentales de la mecánica clásica, como la conservación de la energía mecánica, la relación entre fuerzas y momentos de fuerza, y la influencia del momento de inercia en el movimiento de los cuerpos rígidos. A partir de medidas experimentales de posición, velocidad y tiempo, es posible determinar la aceleración del sistema y obtener una estimación experimental del momento de inercia de la rueda.

\bigskip
En conjunto, la rueda de Maxwell constituye un ejemplo sencillo pero muy completo para analizar cómo se distribuyen y transforman las distintas formas de energía en un sistema físico real, así como para poner en práctica técnicas básicas de medida, análisis gráfico y tratamiento de datos experimentales.

\subsection{Marco Teórico}

Durante la caída de la rueda de Maxwell, el sistema está sometido a dos fuerzas principales: el peso de la rueda, $mg$, dirigido hacia abajo, y la tensión ejercida por los hilos, $T$, dirigida en sentido opuesto. Aplicando la segunda ley de Newton al movimiento de traslación del centro de masas se obtiene

\begin{equation}
m a = m g - T .
\end{equation}

La tensión de los hilos no solo afecta al movimiento traslacional, sino que también origina un momento de fuerza responsable de la rotación de la rueda. Dado que la fuerza es perpendicular al radio en el punto de aplicación, el módulo del torque viene dado por

\begin{equation}
\tau = T R ,
\end{equation}

donde $R$ es el radio de la rueda. De este modo, la segunda ley de Newton para el movimiento de rotación puede expresarse como

\begin{equation}
T R = I \alpha ,
\end{equation}

siendo $I$ el momento de inercia de la rueda respecto a su eje de rotación y $\alpha$ su aceleración angular. 

\bigskip
Suponiendo que no existe deslizamiento entre la rueda y los hilos, la aceleración lineal del centro de masas y la aceleración angular están relacionadas mediante la condición

\begin{equation}
a = \alpha R .
\end{equation}

A partir de esta relación, es posible expresar la tensión de la cuerda en función de la aceleración lineal como

\begin{equation}
T = \frac{I a}{R^{2}} .
\end{equation}

Sustituyendo esta expresión de la tensión en la ecuación del movimiento traslacional, se obtiene finalmente una expresión para la aceleración de caída de la rueda:

\begin{equation}
a = \frac{m g}{m + \frac{I}{R^{2}}} .
\end{equation}

De la expresión obtenida para la aceleración de la rueda se deduce que una medida experimental de dicha magnitud permite determinar su momento de inercia. Para ello, es necesario describir experimentalmente la evolución temporal de la posición y la velocidad del centro de masas. 

\columnbreak

Bajo la hipótesis de aceleración constante, estas magnitudes siguen las leyes del movimiento rectilíneo uniformemente acelerado:

\begin{equation}
x(t) = \frac{1}{2} a t^{2} ,
\end{equation}

\begin{equation}
v(t) = a t .
\end{equation}

En estas expresiones se ha supuesto que la rueda se libera desde el reposo y que la posición inicial coincide con el origen de alturas.

\bigskip
Desde el punto de vista energético, durante la caída la rueda transforma progresivamente su energía potencial gravitatoria en energía cinética. La energía potencial del sistema puede expresarse como

\begin{equation}
E_{p} = - m g z ,
\end{equation}

donde $z$ representa la altura de la rueda respecto al origen elegido. La energía cinética total del sistema está compuesta por dos contribuciones: una asociada al movimiento de traslación del centro de masas y otra debida a la rotación de la rueda alrededor de su eje,

\begin{equation}
E_{c} = \frac{1}{2} m v^{2} + \frac{1}{2} I \omega^{2} .
\end{equation}

Imponiendo de nuevo la condición de no deslizamiento entre la rueda y los hilos, $v = \omega R$, es posible expresar la energía cinética únicamente en función de la velocidad lineal del centro de masas, obteniéndose

\begin{equation}
E_{c} = \frac{1}{2} \left( m + \frac{I}{R^{2}} \right) v^{2} .
\end{equation}

Dado que la rueda se libera desde el reposo y se toma el origen de alturas en su posición inicial, la energía mecánica del sistema en el instante inicial es nula. En estas condiciones, la conservación de la energía mecánica a lo largo del movimiento conduce a la relación

\begin{equation}
\frac{1}{2} \left( m + \frac{I}{R^{2}} \right) v^{2} - m g z = 0 .
\end{equation}

La medida experimental de la posición y la velocidad de la rueda en función del tiempo a lo largo de su trayectoria permite calcular las distintas contribuciones energéticas del sistema y verificar experimentalmente la conservación de la energía mecánica durante la caída.

\subsection{Objetivos}

El objetivo de esta práctica es estudiar experimentalmente el movimiento de la rueda de Maxwell, un sólido rígido que desciende al desenrollarse de unos hilos fijos bajo la acción de la gravedad. Se analizará la evolución de la posición y la velocidad de la rueda en función del tiempo durante la caída, con el fin de caracterizar su movimiento.

\bigskip
A partir de los datos experimentales obtenidos, se comprobará la conservación de la energía mecánica del sistema, estudiando la transformación de la energía potencial gravitatoria en energía cinética de traslación y rotación. Asimismo, el experimento permitirá obtener una estimación del momento de inercia de la rueda y evaluar la coherencia entre los resultados experimentales y el modelo teórico.

\begin{figure}[H]
    \centering
    \includegraphics[width=1\columnwidth]{imagenes/esquemarueda}
    \caption{Esquema del funcionamiento de la rueda de Maxwell}
    \label{fig:esquemarueda}
\end{figure}

\columnbreak

\section{Materiales}

Para la realización del experimento de la rueda de Maxwell se empleó un conjunto de materiales y equipamiento seleccionados con el objetivo de garantizar un montaje experimental estable y reproducible. El uso adecuado de estos elementos resulta fundamental para asegurar la correcta adquisición de los datos experimentales y minimizar las posibles fuentes de error durante las medidas de posición y tiempo.

\bigskip
Los materiales utilizados fueron los siguientes:
\begin{itemize}
  \item Rueda de Maxwell, base e hilos.
  \item Regla.
  \item Barrera fotoeléctrica.
  \item Contador digital.
\end{itemize}

\section{Procedimiento experimental}

Para la realización de la práctica se montó el sistema experimental de la rueda de Maxwell sobre su soporte, asegurando la correcta alineación de los hilos y la estabilidad del conjunto. Antes de iniciar las medidas, se enrolló completamente la rueda alrededor de los hilos y se midieron las magnitudes geométricas necesarias, como las alturas de caída, utilizando una regla graduada.

\subsection{Posición en función del tiempo}

Con el objetivo de determinar la posición de la rueda en función del tiempo, se realizaron medidas del tiempo total de caída para distintas alturas. Para ello, se situó el soporte inicial de la rueda a diferentes posiciones verticales y, en cada caso, se dejó caer la rueda desde el reposo tras desenrollarla completamente. El tiempo de caída se midió mediante el contador digital configurado en modo 1, que registra el intervalo de tiempo transcurrido desde el inicio del movimiento hasta que la rueda atraviesa la barrera fotoeléctrica.

\bigskip
Este procedimiento permite obtener distintos pares de valores altura--tiempo, equivalentes a registrar la evolución de la posición de una única trayectoria en función del tiempo, sin necesidad de realizar un seguimiento continuo del movimiento. Repitiendo las medidas para varias alturas, se obtuvo un conjunto de datos suficiente para caracterizar la cinemática del sistema.

\subsection{Velocidad en función del tiempo}

La determinación de la velocidad de la rueda se llevó a cabo empleando el contador digital en modo 2. En este modo de funcionamiento, el dispositivo mide el tiempo que tarda la rueda en atravesar la barrera fotoeléctrica. Conociendo el diámetro de la barra metálica solidaria a la rueda, es posible calcular la velocidad instantánea del sistema en el punto donde se sitúa la barrera.

\bigskip
Para cada altura considerada en el apartado anterior, se registró el tiempo de paso por la barrera y se calculó la velocidad correspondiente. Cada valor de velocidad se asoció al instante de tiempo determinado previamente para esa posición. De este modo, se pudo reconstruir la evolución temporal de la velocidad de la rueda a lo largo de su caída.

\bigskip
Los datos obtenidos de posición y velocidad en función del tiempo se utilizaron posteriormente para realizar el análisis cinemático y energético del sistema, permitiendo verificar la conservación de la energía mecánica y obtener una estimación experimental del momento de inercia de la rueda.

\columnbreak

\section{Análisis y discusión de datos}

En esta sección se presenta el análisis y la discusión de los datos experimentales obtenidos durante la práctica. Con este fin, se representan las alturas desde las que se libera la rueda en función del tiempo de caída, tanto en escala lineal como en escala logarítmica. El estudio de estas representaciones gráficas permite analizar la relación entre la distancia recorrida y el tiempo.

\begin{figure}[H]
\centering
\includegraphics[width=\linewidth]{imagenes/datos_maxwell}
\caption{Datos experimentales}
\label{fig:datos_maxwell}
\end{figure}

Mediante un ajuste lineal de la representación en escala logarítmica se verifica que la distancia recorrida por la rueda es proporcional al cuadrado del tiempo, de acuerdo con el modelo teórico descrito. A partir de dicho ajuste se determina el valor de la aceleración del sistema, según la ecuación (6). Finalmente, utilizando el valor experimental de la aceleración obtenida, se calcula el momento de inercia de la rueda invirtiendo la ecuación (5).

\bigskip
En la gráfica se aprecia una tendencia lineal clara de los puntos experimentales, siendo las barras de error prácticamente inapreciables debido a la baja incertidumbre de las medidas.

\begin{figure}[H]
\centering
\includegraphics[width=1\columnwidth]{imagenes/grafica1}
\caption{}
\label{fig:grafica1}
\end{figure}

En esta otra gráfica se muestra la representación logarítmica de la altura en función del tiempo de caída. El ajuste lineal obtenido en esta escala confirma la dependencia cuadrática de la distancia con el tiempo, característica de un movimiento uniformemente acelerado.

\begin{figure}[H]
\centering
\includegraphics[width=\linewidth]{imagenes/grafica2}
\caption{}
\label{fig:grafica2}
\end{figure}

A partir de los ajustes realizados sobre las representaciones gráficas de los datos experimentales, se obtuvieron los parámetros característicos del movimiento de la rueda de Maxwell. En particular, el ajuste logarítmico permitió determinar la aceleración de caída del sistema y, a partir de ella, calcular el momento de inercia de la rueda, junto con sus correspondientes incertidumbres.

\begin{figure}[H]
\centering
\includegraphics[width=0.8\columnwidth]{imagenes/datosgraficas12}
\caption{}
\label{fig:datosgrafica}
\end{figure}

El análisis de los datos experimentales muestra una buena concordancia con el modelo teórico del M.R.U.A. En la representación de la altura en función del tiempo se observa una tendencia creciente clara, mientras que la representación logarítmica de la altura frente al tiempo permite comprobar de forma más precisa la relación cuadrática entre ambas magnitudes.

\columnbreak

\bigskip
El ajuste lineal realizado sobre la representación en escala log--log muestra un coeficiente de determinación elevado, lo que indica una buena concordancia entre los datos experimentales y el modelo teórico propuesto. Aunque la pendiente obtenida no coincide exactamente con el valor teórico esperado de 2, la discrepancia observada es relativamente pequeña y puede atribuirse a diversos efectos no ideales del experimento, como la presencia de rozamientos en el eje de la rueda, pérdidas de energía durante el desenrollado de los hilos o pequeñas imprecisiones en la medición de los tiempos de caída.

\bigskip
A partir del ajuste se obtuvo un valor de la aceleración de caída de la rueda que resulta compatible, dentro de su incertidumbre, con un sistema en el que coexisten el movimiento de traslación del centro de masas y la rotación del sólido rígido. El momento de inercia calculado a partir de este valor experimental presenta un orden de magnitud coherente con el esperado para la geometría de la rueda utilizada, y su incertidumbre relativa es reducida, lo que respalda la fiabilidad del procedimiento experimental seguido.

\bigskip
En conjunto, los resultados obtenidos presentan un comportamiento físicamente consistente y se encuentran en buen acuerdo con las predicciones teóricas dentro de los márgenes de error considerados. Esto permite validar tanto el planteamiento experimental como el método de análisis aplicado en esta práctica.

\bigskip
Una vez completado el estudio de la posición de la rueda en función del tiempo, se procede a analizar de manera independiente su comportamiento dinámico a través de la velocidad instantánea. Este segundo enfoque proporciona una determinación alternativa de la aceleración del sistema y, en consecuencia, del momento de inercia de la rueda, permitiendo comparar ambos métodos y evaluar su consistencia.

\columnbreak

En la siguiente gráfica se representa la velocidad instantánea de la rueda en función del tiempo de caída, obtenida a partir de las medidas del contador digital en modo 2. Los datos experimentales muestran un comportamiento aproximadamente lineal, característico de un movimiento uniformemente acelerado.

\begin{figure}[H]
\centering
\includegraphics[width=\linewidth]{imagenes/grafica3}
\caption{}
\label{fig:grafica3}
\end{figure}

A partir del ajuste lineal de la velocidad instantánea en función del tiempo de caída, se obtuvo una nueva estimación de la aceleración del sistema. Utilizando este valor y aplicando el correspondiente análisis de propagación de errores, fue posible determinar el momento de inercia de la rueda. De este modo, se obtuvieron los siguientes resultados experimentales:

\begin{figure}[H]
\centering
\includegraphics[width=0.7\columnwidth]{imagenes/datos3}
\caption{}
\label{fig:datos3}
\end{figure}

Para obtener los resultados presentados se parte de las medidas del contador digital en modo 2, que proporcionan el tiempo $t_{\mathrm{inst}}$ que la barra solidaria a la rueda tarda en atravesar la barrera fotoeléctrica. Conociendo el diámetro de la barra, la velocidad instantánea se calcula como
\begin{equation}
v = \frac{d}{t_{\mathrm{inst}}}.
\end{equation}
Cada valor de velocidad se asocia al instante de tiempo correspondiente de la caída, lo que permite representar la velocidad en función del tiempo.

\bigskip
Suponiendo que el movimiento es uniformemente acelerado, la ecuación (7) predice una relación lineal entre la velocidad y el tiempo,
\begin{equation}
v(t) = v_0 + a t,
\end{equation}
de modo que un ajuste lineal de la gráfica $v(t)$ permite obtener la aceleración $a$ a partir de la pendiente de la recta. Este valor puede compararse con el obtenido en el apartado anterior a partir de la ecuación (6), basada en la evolución temporal de la posición.

bigskip
La incertidumbre de la velocidad se ha estimado mediante propagación de errores, teniendo en cuenta las incertidumbres asociadas al diámetro de la barra y al tiempo medido por el contador. A partir del ajuste lineal se obtiene el valor de la aceleración con su error, y posteriormente el momento de inercia se calcula invirtiendo la ecuación (5). La incertidumbre del momento de inercia se determina también por propagación, considerando los errores en la masa, el radio y la aceleración.

\bigskip
En cuanto a la discusión de los resultados, la gráfica $v(t)$ muestra un incremento de la velocidad con el tiempo, coherente con un movimiento acelerado. No obstante, el valor de la aceleración obtenido por este método resulta significativamente menor que el determinado en el apartado anterior a partir de la posición, y ambos valores no coinciden dentro de sus respectivas incertidumbres. Esta discrepancia puede atribuirse a la mayor sensibilidad del método basado en la velocidad a errores sistemáticos, especialmente a la incertidumbre en el diámetro efectivo de la barra y a la medida del tiempo instantáneo.

\bigskip
Finalmente, al comparar la precisión de ambos métodos, se observa que el procedimiento basado en la posición (ecuación (6)) presenta una incertidumbre relativa considerablemente menor que el basado en la velocidad. Por tanto, puede concluirse que los resultados obtenidos a partir de la posición son más precisos, mientras que el método de la velocidad proporciona una estimación menos fiable del momento de inercia debido a la acumulación de errores experimentales.

\bigskip
Una vez caracterizado el movimiento de la rueda desde el punto de vista cinemático y determinadas experimentalmente la aceleración y el momento de inercia, se procede a analizar el sistema desde el punto de vista energético. Este estudio permite comprobar de manera directa si la energía mecánica se conserva a lo largo de la caída de la rueda.

\bigskip
Conociendo la posición de la rueda en función del tiempo, se calcula la energía potencial gravitatoria del sistema en cada instante mediante la ecuación (8). De forma análoga, se determina la energía cinética de la rueda en función del tiempo utilizando la ecuación (10). La suma de ambas contribuciones permite obtener la energía mecánica total del sistema a lo largo de la trayectoria de caída.

\bigskip
Finalmente, se representan gráficamente la energía potencial, la energía cinética y la energía mecánica en función del tiempo, y se analiza si los resultados experimentales evidencian la conservación de la energía mecánica durante el movimiento.

\begin{figure}[H]
\centering
\includegraphics[width=\linewidth]{imagenes/grafica5}
\caption{Energías potencial, cinética y mecánica de la rueda en función del tiempo}
\label{fig:grafica5}
\end{figure}

Si se observa la gráfica de las energías en función del tiempo y se analizan los valores medios de la energía potencial, cinética y mecánica, se aprecia que, aunque la energía potencial disminuye y la energía cinética aumenta durante la caída, la energía mecánica total permanece aproximadamente constante. Dentro de las incertidumbres experimentales, este comportamiento indica que existe conservación de la energía mecánica en el sistema.

\begin{figure}[H]
\centering
\includegraphics[width=\linewidth]{imagenes/medias}
\caption{Valores medios de las energías potencial, cinética y mecánica obtenidos a partir de los datos experimentales}
\label{fig:medias}
\end{figure}

\begin{figure}[H]
\centering
\includegraphics[width=0.75\linewidth]{imagenes/ultimosdatos}
\caption{Resultados finales obtenidos a partir del análisis energético del sistema}
\label{fig:ultimosdatos}
\end{figure}

A partir del análisis de la evolución temporal de las energías y del estudio de sus valores medios, se observa que la energía mecánica del sistema se mantiene aproximadamente constante a lo largo de la caída de la rueda. Aunque la energía potencial gravitatoria disminuye progresivamente y la energía cinética aumenta de forma correspondiente, la suma de ambas no presenta variaciones significativas dentro de las incertidumbres experimentales. Por tanto, puede concluirse que, en el marco de precisión de la práctica y pese a posibles pérdidas por rozamiento, se observa la conservación de la energía mecánica del sistema.

\columnbreak

\section{Conclusiones}

En esta práctica se ha estudiado experimentalmente el movimiento de la rueda de Maxwell, analizando tanto su comportamiento cinemático como energético. A partir de la medida de la posición en función del tiempo se comprobó que el movimiento de la rueda puede describirse como un movimiento uniformemente acelerado, en concordancia con el modelo teórico. El ajuste de los datos experimentales permitió determinar la aceleración del sistema y, a partir de ella, obtener una estimación del momento de inercia de la rueda.

\bigskip
Asimismo, se llevó a cabo un segundo análisis basado en la medida de la velocidad instantánea mediante una barrera fotoeléctrica. Aunque este método permitió obtener de forma independiente un valor de la aceleración, se observó que los resultados presentaban una mayor incertidumbre y no coincidían plenamente con los obtenidos a partir del análisis de la posición. Esta discrepancia pone de manifiesto la mayor sensibilidad del método basado en la velocidad a errores sistemáticos e instrumentales, especialmente relacionados con la medida del diámetro efectivo de la barra y del tiempo de paso por la barrera.

\bigskip
El estudio energético del sistema mostró que, durante la caída de la rueda, la energía potencial gravitatoria se transforma progresivamente en energía cinética de traslación y de rotación. A partir de la representación de las energías en función del tiempo y del análisis de sus valores medios, se comprobó que la energía mecánica total se mantiene aproximadamente constante dentro de las incertidumbres experimentales, lo que indica la conservación de la energía mecánica en el sistema, pese a la posible presencia de pequeñas pérdidas por rozamiento.

\bigskip
En conjunto, los resultados obtenidos son coherentes con el modelo teórico de la rueda de Maxwell y permiten ilustrar de forma experimental conceptos fundamentales de la mecánica clásica, como el movimiento uniformemente acelerado, la relación entre traslación y rotación y la conservación de la energía. La práctica pone de manifiesto, además, la importancia del tratamiento riguroso de los errores experimentales y de la elección adecuada del método de medida para obtener resultados precisos y fiables.

\end{multicols}

\end{document}